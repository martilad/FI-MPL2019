%preamble declarations
\tolerance=500
%\raggedbottom
\usepackage{a4wide}
%\parskip=0pt
%\parindent=0pt
%\setlength{\parindent}{0pt}
%\setlength{\parskip}{1ex}
\setcounter{secnumdepth}{3}
\setcounter{tocdepth}{3}
%\def\clqq{\vbox to0pt{\vss\hbox{\char34}\kern-1.4ex}}
%\chardef\crqq=92
%\def\uv{\bgroup\aftergroup\closequotes
%\leavevmode\clqq\let\next=}
%\def\closequotes{\unskip\crqq\relax}
\def\uv#1{``#1''}
\def\whiten{}


\def\ifndef#1{\expandafter\ifx\csname#1\endcsname\relax} % check whether macro is defined

%inserting figures either in pdf or eps
\newcommand\cobrx[2]
 {\begin{center}
  \ifndef{pdfpagewidth} % check if PDFLaTeX is processing this document
    % if not then include EPS figures
    \includegraphics*[width=#1]{#2.eps}\else
    % otherwise include PDF figures
    \includegraphics*[width=#1]{#2.pdf}
   \fi
  \end{center}}

\newcommand\coverpagestarts
{\pagenumbering{arabic}\thispagestyle{empty}
\begin{center}
\large\sffamily
\University\\
\Faculty\\
\vglue 10mm
\cobrx{50mm}{img/LogoCVUT}
\vglue 30mm
{\large\bfseries\Title}\\
\bigskip
\bigskip
{\large\emph{\Name}}\\
\vfill
\vglue 20mm
\Department\\
\Program\\

\bigskip
\today
\end{center}
\ifndef{CoSupervisor}\relax\else\bigskip
\begin{description}
\item[Co-Supervisor:]\ \\
\CoSupervisor\\
\CoSupervisorAffiliation\\
\end{description}
\vglue 1cm
\fi

}

\newcommand\mainbodystarts{
\pagestyle{headings}\pagenumbering{arabic}\setcounter{page}{1}
}

%include any longer words that by default hyphenate improperly
\hyphenation{di-men-sion-al ma-te-ma-ti-ky hard-ware trans-fer-able}

%Chapters start at odd pages if you include:
%\newcommand\chapter{\if@openright\cleardoublepage\else\clearpage\fi
% into /usr/lib/texmf/tex/latex/latex/book.cls

%your specific macros

%structuring of mathematical texts
\newtheorem{remark}{Remark}[section]
\newtheorem{preposition}{Preposition}[section]
\newtheorem{assumption}{Assumption}[section]
\newtheorem{lemma}{Lemma}[section]
\newtheorem{note}{Note}[section]
\newtheorem{definition}{Definition}[section]
\newtheorem{example}{Example}[section]
\newtheorem{corollary}{Corollary}[section]
\newtheorem{theorem}{Theorem}[section]
\newtheorem{proposition}{Proposition}[section]
%
\newcommand{\bcen}{\begin{center}}
\newcommand{\ecen}{\end{center}}
\newcommand{\blem}{\begin{lemma}\sl}
\newcommand{\elem}{\end{lemma}\rm}
\newcommand{\bnote}{\begin{note}\rm}
\newcommand{\enote}{\end{note}}
\newcommand{\bcor}{\begin{corollary}\sl}
\newcommand{\ecor}{\end{corollary}\rm}
\newcommand{\bdefi}{\begin{definition}\rm}
\newcommand{\edefi}{\end{definition}}
\newcommand{\btheo}{\begin{theorem}\sl}
\newcommand{\etheo}{\end{theorem}\rm}
\newcommand{\bprop}{\begin{proposition}\sl}
\newcommand{\eprop}{\end{proposition}\rm}
\newcommand{\bexam}{\begin{example}\rm}
\newcommand{\eexam}{\end{example}}
%
\newcommand{\bfig}{\begin{figure}\begin{center}}
\newcommand{\efig}{\end{center}\end{figure}}
\newcommand{\btab}{\begin{table}\begin{center}}
\newcommand{\etab}{\end{center}\end{table}}
\newcommand{\benum}{\begin{enumerate}}
\newcommand{\eenum}{\end{enumerate}}
\newcommand{\bitem}{\begin{itemize}}
\newcommand{\eitem}{\end{itemize}}
\newcommand{\bflushr}{\begin{flushright}}
\newcommand{\eflushr}{\end{flushright}}


\def\TOPRbox#1#2{\setbox0\hbox{\framebox{#2}}\vbox{\hbox to\wd0{\hss\hbox{#1}\hss}\kern 2pt\hbox to\wd0{\hss\hbox{\rotatebox{90}{\hbox{$=$}}}\hss}\kern 2pt\copy0}}
\def\BOTRbox#1#2{\setbox0\hbox{\framebox{#2}}\vbox{\copy0\kern 2pt\hbox to\wd0{\hss\hbox{\rotatebox{90}{\hbox{$=$}}}\hss}\kern 2pt\hbox to\wd0{\hss\hbox{#1}\hss}}}
\def\eqc#1{%
\setbox0\hbox{$=$}%
\setbox1\hbox to\wd0{\hss$\downarrow$\hss}%
\setbox3\hbox{\framebox{\vbox{#1}}}
\setbox2\hbox to\wd0{\hss\box3\hss}
\vbox{\box2\kern 5pt\box1\box0}}

%mark for the end of a proof or theorem
\def\sq{\hfill\hbox{\rlap{$\sqcap$}$\sqcup$}\vskip \bigskipamount}

\def\flcaption#1{\figurename{ }\addtocounter{figure}{1}\thefigure: #1%
\addcontentsline{lof}{figure}{\numberline {\thefigure}{\ignorespaces #1}}}
\def\tbcaption#1{\tablename{ }\addtocounter{table}{1}\thetable: #1%
\addcontentsline{lot}{table}{\numberline {\thetable}{\ignorespaces #1}}}

\def\norm#1{\left|\left|#1\right|\right|}
\def\MRC{\mathop{\mathrm{MRC}}\nolimits}
\def\swap{\mathop{\mathrm{swap}}\nolimits}
\def\erf{\mathop{\mathrm{erf}}\nolimits}
\def\rank{\mathop{\mathrm{rank}}\nolimits}
\def\cond{\mathop{\mathrm{cond}}\nolimits}
\def\diag{\mathop{\mathrm{diag}}\nolimits}
\def\cov{\mathop{\mathrm{cov}}\nolimits}
\def\fl{\mathop{\mathrm{fl}}\nolimits}
\def\ex{\mathop{\mathrm{ex}}\nolimits}
\def\even{\mathop{\mathrm{even}}\nolimits}
\def\round{\mathop{\mathrm{round}}}

\def\Thesis{Report}
\def\thesis{report}


\newenvironment{proof}{\par\bigskip\noindent{\bf Proof.} }{\hbox{ }\hfill$\square$\par\bigskip\par}
%
%mark for the end of a proof or theorem
\def\sq{\hfill\hbox{\rlap{$\sqcap$}$\sqcup$}\vskip \bigskipamount}
\def\rulepar#1{\noindent\setbox1\hbox{{\bf{#1}}}%
\newdimen\sirka\sirka=\wd1%
\newdimen\rsirka\rsirka=0.5\textwidth%
\advance\rsirka by-40pt%
\setbox2\hbox to\rsirka{\hss\vbox to\ht1{\vss\hrule width \rsirka\vss\hrule width \rsirka\vss\hrule width \rsirka\vss}\hss}%
\leavevmode\hbox to\hsize{\hss\copy2\hss\copy1\hss\copy2\hss}}

%%--------------------------------------------------------------------------------
%% The registered trademark symbol, a raised, circled R.  We do a little
%% negative kern on each side because the edge of the circle is quite far from
%% the corner of the box.
%%
%% Scott D. Anderson (anderson@cs.umass.edu)
%%
\def\Registered{\raisebox{1ex}{\kern-.1em\setbox\@tempboxa\hbox{\footnotesize$\bigcirc$}\hbox to 0pt{\hbox to\wd\@tempboxa{\hss\kern-.5pt\tiny R\hss}\hss}\box\@tempboxa\kern-.1em}}
%%--------------------------------------------------------------------------------

%%%
\def\myplus{\scalebox{0.6}{\begin{picture}(10,10)(-5,-5)\thinlines
\put(-5,0){\line(1,0){10}}
\put(0,5){\line(0,-1){10}}
\end{picture}}}
%%%
\def\mysquare{\scalebox{0.6}{\begin{picture}(10,10)(-5,-5)\thinlines
\put(-5,-5){\line(1,0){10}}
\put(5,-5){\line(0,1){10}}
\put(5,5){\line(-1,0){10}}
\put(-5,5){\line(0,-1){10}}
\end{picture}}}
%%%
\def\mystar{\scalebox{0.6}{\begin{picture}(10,10)(-5,-5)\thinlines
\put(-5,0){\line(1,0){10}}
\put(0,5){\line(0,-1){10}}
\put(-5,-5){\line(1,1){10}}
\put(-5,5){\line(1,-1){10}}
\end{picture}}}
%%%
\def\mycross{\scalebox{0.6}{\begin{picture}(10,10)(-5,-5)\thinlines
\put(-5,-5){\line(1,1){10}}
\put(-5,5){\line(1,-1){10}}
\end{picture}}}

\def\eqref#1{(\ref{#1})}
\def\tabref#1{Tab.~\ref{#1}}
\def\figref#1{Fig.~\ref{#1}}
\def\algref#1{Alg.~\ref{#1}}

\def\text#1{#1}
